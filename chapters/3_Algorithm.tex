\chapter{\APFD{}的\Distribution}

为了分析指标\APFD{}的\distribution{}。
我们首先提出一个算法来计算一般情况的$M$下的\APFD{}的\distribution{},然后我们讨论两个特殊情况,即最近RTP研究中最常见的\mappingAllToOne{}\mappingMatrices{}和\mappingOneToOne{}\mappingMatrices{}。

\section{一般情况下计算\APFD{}的\Distribution{}的算法}\label{sec:dpalgorithm}

为了计算\APFD{}的\distribution{},一个朴素的算法会将列举所有$n!$测试用例执行次序,并为每个执行次序计算\APFD{}。
理论上,\APFD{}最多可以取$O(n!)$不同的值,例如,当$m=\sum_{i=1}^nn^i$并且所有$n$个测试用例失败,并且分别\detect 到$n,n^2,\ldots,n^n$个不同的缺陷时,$n!$中的每个执行次序都有一个不同的\APFD{}。
然而,在实践中,缺陷的数量$m$和失败的数量$k$通常很小,例如,在我们的实验数据集~\cite{Peng2020IRTCP}中,2980个(98\%)\job{}中的2906个有$k\le 10$。
我们提出了一种算法,以$O(n^2mk\multi k!)$的时间复杂度计算精确的\distribution{}。
尽管有其中包含$k!$的项,该算法在实践中的运行时间是合理的,对于2906个\job{}中的任何一个都在30秒以内运行结束。
当$k>10$时,我们可以采用抽样的方法。

接下来我们描述一下我们算法的直觉。
$sum_{i=1}^m\TF_i$是\APFD{}中唯一依赖于测试用例执行次序的项,所以我们首先计算这个项的分布,然后将其转换为\APFD{}的分布。
枚举所有的缺陷求和并不能为其带来一个好的递归公式。
我们算法的关键是,将其转化为\emph{枚举$k$个失败测试用例的位置求和}。
我们把$sum_{i=1}^m\TF_i$看作是一个\emph{加权和}。
\begin{equation}\label{TF:weighted:sum}
\sum_{i=1}^m\TF_i=\sum_{j=1}^kw_j\position_j
\end{equation}
其中$\position_j$是执行次序中第j个失败测试用例的位置,$w_j\ge 0$是权重,其计算方式为第$j$个失败测试用例检测到的缺陷数量(算法~\ref{algo:distribution:APFD}的第~\ref{w_i:define}行)。
例如,考虑一下图~\ref{fig:example}中的测试用例执行次序$o1$。
$k=4$个失败测试用例的\emph{相对顺序}是$\relativeOrder=\langle\mathrm{C2.t2,C1.t1,C1.t3,C2.t1}\rangle$;我们使用记号$\relativeOrder$来区别于所有$n$个测试用例的执行次序的记号$o$。
对于这个相对顺序,$w=\langle 1,1,1,0\rangle$,因为$m=3$个缺陷首先由$\mathrm{C2.t2}$、$\mathrm{C1.t1}$和$\mathrm{C1.t3}$检测到。
对应这个相对顺序$\relativeOrder$的$\position$是$\position=\langle 1,2,3,5\rangle$,因为$\relativeOrder$中的$4$个失败测试用例在执行次序$o1$中出现在这些位置。

对于$\position=\langle\position_1\ldots\position_k\rangle$,如果$1\le\position_1<\ldots<\position_k\le n$,我们称$\position$是\emph{合法}的。
$\position$和$w=\langle w_1\ldots w_k\rangle$都可以在不同的测试用例执行次序中变化。
虽然$\position$有$\binom{n}{k}$个有效值,但因为$w$只取决于$\relativeOrder$,我们注意到$w$最多有$k!$个可能值(在实践中$k!\ll \binom{n}{k}$,因为$k\ll n$)。
因此,我们首先通过枚举$k$个失败测试用例的$k!$种相对顺序来固定$w$。
然后对于每种相对顺序,计算$\sum_{i=1}^m\TF_i=\sum_{j=1}^kw_j\position_j$的\distribution{}问题等价于\emph{“给定$w$,计算对于每个$s$使得$\sum_{j=1}^kw_j\position_j=s$的有效$\position$的数量"}。这个问题这可以按以下方式递归解决。

我们用$f_{\paramOneInF,\paramTwoInF}(s)$表示满足$1\le \position_1< \ldots < \position_{\paramOneInF}\le \paramTwoInF$且$\sum_{j=1}^{\paramOneInF}w_j\position_j=s$的情况下,$\position_1,\ldots,\position_{\paramOneInF}$的可能取值数量。那么我们要解决的问题就是求出$f_{k,n}(s)$。
作为基础情况,
(1)~$\paramOneInF>\paramTwoInF$时$f_{\paramOneInF,\paramTwoInF}(s)=0$,因为$\position_{\paramOneInF}<\paramOneInF$不可能成立。
(2)~$f_{0,\paramTwoInF}(s)=\mathbf{1}_{s=0}$,其中$\mathbf{1}$是指示函数,因为只有空序列$\langle\rangle$是合法的,并且$\sum_{j=1}^0w_j\position_j=0$。
对于所有$\paramTwoInF\ge \paramOneInF>0$,$f_{\paramOneInF,\paramTwoInF}(s)$的可能取值数有两种情况。(1)~如果$\position_{\paramOneInF}\le \paramTwoInF-1$,根据定义,这个数等于$f_{\paramOneInF,\paramTwoInF-1}(s)$。(2)~如果$\position_{\paramOneInF}=\paramTwoInF$,$s$的可能数量等于满足$\position_{\paramOneInF-1}\le \position_{\paramOneInF}-1= \paramTwoInF-1$和$\sum_{j=1}^{\paramOneInF-1}w_j\position_j=(\sum_{j=1}^{\paramOneInF}w_j\position_j)-w_{\paramOneInF}\position_{\paramOneInF}=s-w_{\paramOneInF}\paramTwoInF$的情况下$\position_1,\ldots,\position_{\paramOneInF-1}$的可能取值数,也就是$f_{\paramOneInF-1,\paramTwoInF-1}(s- w_{\paramOneInF}\paramTwoInF)$。
总的来说,我们有
\begin{equation}\label{f:recursive:equation}
f_{\paramOneInF,\paramTwoInF}(s)=\left\{
\begin{array}{lc}
0 & \quad\paramOneInF>\paramTwoInF\\
\mathbf{1}_{s=0} & \quad\paramOneInF=0\\
f_{\paramOneInF,\paramTwoInF-1}(s)+f_{\paramOneInF-1,\paramTwoInF-1}(s-w_{\paramOneInF}\paramTwoInF) & \quad\mathrm{其它情况}
\end{array}
\right.
\end{equation}
在解出$f_{k,n}$之后,我们得到了对于每种失败测试用例的相对位置的$\sum_{i=1}^m\TF_i$的\distribution{}。
根据对称性,$k!$种相对顺序都有相同的概率取到,因此我们只需这些\distributions{}的平均,就可以得到对所有测试用例执行次序的$\sum_{i=1}^mTF_i$的\distribution{}。
最后,我们将$\sum_{i=1}^mTF_i$的\distribution{}转换为\APFD{}的\distribution{}。

\begin{algorithm}[t]
\renewcommand\baselinestretch{0.9}\selectfont
\caption{计算\APFD{}的\distribution}
\label{algo:distribution:APFD}
\textbf{Input:} n,m,M \ccc{测试用例和缺陷的数量, 以及\mappingMatrix{}}\\

\textbf{Output:} $\Output$ \ccc{\APFD{}的\distribution{}: $\Output(x)=\Prob{\APFD{}=x}$}\\

\SetKwFunction{FMain}{PMF}
\SetKwProg{Fn}{Function}{}{}
\Fn{\FMain{}\emph{\ccc{主函数; 返回\APFD{}对所有测试用例执行次序的\distribution{}}}}{
    k = $\mathrm{|\{j|\exists i.\twoParaFunc{M}{j}{i}\}|}$ \ccc{M中失败的数量, 实践中k$\ll$n}\\
    q = \FuncSty{PMF\_sum()} \ccc{计算$\sum_{i=1}^m\TF_i$的\distribution}\\
    \KwRet{$\mathrm{\lambda x.q(mn-mnx+\frac{m}{2})}$} \ccc{将$\sum_{i=1}^m\TF_i$的\distribution{}转化为\APFD{}的\distribution{}}
}

\SetKwFunction{FMain}{PMF\_sum}
\SetKwProg{Fn}{Function}{}{}
\Fn{\FMain{}\emph{\ccc{return the \distribution of $\mathrm{\sum_{i=1}^m\TF_i}$ for all orders}}}{
    $\mathcal{P}$ = $\langle$\FuncSty{PMF\_rorder($\relativeOrder$)}$,\forall \relativeOrder \in perms(\{j|\exists i.\twoParaFunc{M}{j}{i}\})\rangle$ \ccc{枚举所有的相对位置}\label{line:enumerate:o}\\
    \KwRet{$\mathrm{\lambda x.\sum_{p\in \mathcal{P}}p(x)/|\mathcal{P}|}$} \ccc{将每种相对位置的$\sum_{i=1}^m\TF_i$的\distributions{}做平均}
}

\SetKwFunction{FMain}{PMF\_rorder}
\SetKwProg{Fn}{Function}{}{}
\Fn{\FMain{$\mathrm{\relativeOrder}$}\emph{\ccc{返回对于一种相对位置$\mathrm{\relativeOrder}$,$\mathrm{\sum_{i=1}^m\TF_i}$的\distribution{}}}}{
    w = $\langle | \{ i | \twoParaFunc{M}{\relativeOrder_j}{i}  \wedge \nexists j'<j. \twoParaFunc{M}{\relativeOrder_{j'}}{i}\} |, \forall j \in 1..k \rangle$
    \ccc{$\mathrm{w}$是公式\ref{TF:weighted:sum}中的权重}\label{w_i:definition}
    \\
    \KwRet{$\lambda s.$\FuncSty{f($\mathrm{w,k,n}$)}$(s)/\binom{n}{k}$} \ccc{$\position$的总共可能取值有$\binom{n}{k}$个}
}

\ccc{下面的函数应该被记忆化,以避免对重复的w,\paramOneInF,和\paramTwoInF{}组合重复计算}\\

\SetKwFunction{FMain}{f}
\SetKwProg{Fn}{Function}{}{}
\Fn{\FMain{$\mathrm{w,\paramOneInF,\paramTwoInF}$}\emph{\ccc{给定w,用公式\ref{f:recursive:equation}计算并返回$\mathrm{f_{\paramOneInF,\paramTwoInF}}$}}}{ 
    \uIf{$\mathrm{\paramOneInF>\paramTwoInF}$}{
        \KwRet{$\mathrm{\lambda s.0}$}
    }
    \uIf{$\mathrm{\paramOneInF=0}$}{
        \KwRet{$\mathrm{\lambda s.\mathbf{1}_{s=0}}$}
    }
    \KwRet{$\lambda s.$\FuncSty{f($\mathrm{w,\paramOneInF,\paramTwoInF-1}$)}$\mathrm{(s)+}$\FuncSty{f($\mathrm{w,\paramOneInF-1,\paramTwoInF-1}$)}$\mathrm{(s-w_{\paramOneInF}\paramTwoInF)}$}
}
\end{algorithm}
