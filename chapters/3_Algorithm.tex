\chapter{\APFD{}的\Distribution}

为了分析指标\APFD{}的\distribution{}。
我们首先提出一个算法来计算一般情况的$M$下的\APFD{}的\distribution{},然后我们讨论两个特殊情况,即最近RTP研究中最常见的\mappingAllToOne{}\mappingMatrices{}和\mappingOneToOne{}\mappingMatrices{}。

\section{一般情况下计算\APFD{}的\Distribution{}的算法}\label{sec:dpalgorithm}

为了计算\APFD{}的\distribution{},一个朴素的算法会将列举所有$n!$测试用例执行次序,并为每个执行次序计算\APFD{}。
理论上,\APFD{}最多可以取$O(n!)$不同的值,例如,当$m=\sum_{i=1}^nn^i$并且所有$n$个测试用例失败,并且分别\detect 到$n,n^2,\ldots,n^n$个不同的缺陷时,$n!$中的每个执行次序都有一个不同的\APFD{}。
然而,在实践中,缺陷的数量$m$和失败的数量$k$通常很小,例如,在我们的实验数据集~\cite{Peng2020IRTCP}中,2980个(98\%)\job{}中的2906个有$k\le 10$。
我们提出了一种算法,以$O(n^2mk\multi k!)$的时间复杂度计算精确的\distribution{}。
尽管有其中包含$k!$的项,该算法在实践中的运行时间是合理的,对于2906个\job{}中的任何一个都在30秒以内运行结束。
当$k>10$时,我们可以采用抽样的方法。

接下来我们描述一下我们算法的直觉。
$sum_{i=1}^m\TF_i$是\APFD{}中唯一依赖于测试用例执行次序的项,所以我们首先计算这个项的分布,然后将其转换为\APFD{}的分布。
枚举所有的缺陷求和并不能为其带来一个好的递归公式。
我们算法的关键是,将其转化为\emph{枚举$k$个失败测试用例的位置求和}。
我们把$sum_{i=1}^m\TF_i$看作是一个\emph{加权和}。
\begin{equation}\label{TF:weighted:sum}
\sum_{i=1}^m\TF_i=\sum_{j=1}^kw_j\position_j
\end{equation}
其中$\position_j$是执行次序中第j个失败测试用例的位置,$w_j\ge 0$是权重,其计算方式为第$j$个失败测试用例检测到的缺陷数量(算法~\ref{algo:distribution:APFD}的第~\ref{w_i:define}行)。
例如,考虑一下图~\ref{fig:example}中的测试用例执行次序$o1$。
$k=4$个失败测试用例的\emph{相对顺序}是$\relativeOrder=\langle\mathrm{C2.t2,C1.t1,C1.t3,C2.t1}\rangle$;我们使用记号$\relativeOrder$来区别于所有$n$个测试用例的执行次序的记号$o$。
对于这个相对顺序,$w=\langle 1,1,1,0\rangle$,因为$m=3$个缺陷首先由$\mathrm{C2.t2}$、$\mathrm{C1.t1}$和$\mathrm{C1.t3}$检测到。
对应这个相对顺序$\relativeOrder$的$\position$是$\position=\langle 1,2,3,5\rangle$,因为$\relativeOrder$中的$4$个失败测试用例在执行次序$o1$中出现在这些位置。

对于$\position=\langle\position_1\ldots\position_k\rangle$,如果$1\le\position_1<\ldots<\position_k\le n$,我们称$\position$是\emph{合法}的。
$\position$和$w=\langle w_1\ldots w_k\rangle$都可以在不同的测试用例执行次序中变化。
虽然$\position$有$\binom{n}{k}$个有效值,但因为$w$只取决于$\relativeOrder$,我们注意到$w$最多有$k!$个可能值(在实践中$k!\ll \binom{n}{k}$,因为$k\ll n$)。
因此,我们首先通过枚举$k$个失败测试用例的$k!$种相对顺序来固定$w$。
然后对于每种相对顺序,计算$\sum_{i=1}^m\TF_i=\sum_{j=1}^kw_j\position_j$的\distribution{}问题等价于\emph{“给定$w$,计算对于每个$s$使得$\sum_{j=1}^kw_j\position_j=s$的有效$\position$的数量"}。这个问题这可以按以下方式递归解决。

\begin{algorithm}[t]
\renewcommand\baselinestretch{0.9}\selectfont
\caption{计算\APFD{}的\distribution}
\label{algo:distribution:APFD}
\textbf{Input:} n,m,M \ccc{测试用例和缺陷的数量, 以及\mappingMatrix{}}\\

\textbf{Output:} $\Output$ \ccc{\APFD{}的\distribution{}: $\Output(x)=\Prob{\APFD{}=x}$}\\

\SetKwFunction{FMain}{PMF}
\SetKwProg{Fn}{Function}{}{}
\Fn{\FMain{}\emph{\ccc{主函数; 返回\APFD{}对所有测试用例执行次序的\distribution{}}}}{
    k = $\mathrm{|\{j|\exists i.\twoParaFunc{M}{j}{i}\}|}$ \ccc{M中失败的数量, 实践中k$\ll$n}\\
    q = \FuncSty{PMF\_sum()} \ccc{计算$\sum_{i=1}^m\TF_i$的\distribution}\\
    \KwRet{$\mathrm{\lambda x.q(mn-mnx+\frac{m}{2})}$} \ccc{将$\sum_{i=1}^m\TF_i$的\distribution{}转化为\APFD{}的\distribution{}}
}

\SetKwFunction{FMain}{PMF\_sum}
\SetKwProg{Fn}{Function}{}{}
\Fn{\FMain{}\emph{\ccc{return the \distribution of $\mathrm{\sum_{i=1}^m\TF_i}$ for all orders}}}{
    $\mathcal{P}$ = $\langle$\FuncSty{PMF\_rorder($\relativeOrder$)}$,\forall \relativeOrder \in perms(\{j|\exists i.\twoParaFunc{M}{j}{i}\})\rangle$ \ccc{枚举所有的相对位置}\label{line:enumerate:o}\\
    \KwRet{$\mathrm{\lambda x.\sum_{p\in \mathcal{P}}p(x)/|\mathcal{P}|}$} \ccc{将每种相对位置的$\sum_{i=1}^m\TF_i$的\distributions{}做平均}
}

\SetKwFunction{FMain}{PMF\_rorder}
\SetKwProg{Fn}{Function}{}{}
\Fn{\FMain{$\mathrm{\relativeOrder}$}\emph{\ccc{返回对于一种相对位置$\mathrm{\relativeOrder}$,$\mathrm{\sum_{i=1}^m\TF_i}$的\distribution{}}}}{
    w = $\langle | \{ i | \twoParaFunc{M}{\relativeOrder_j}{i}  \wedge \nexists j'<j. \twoParaFunc{M}{\relativeOrder_{j'}}{i}\} |, \forall j \in 1..k \rangle$
    \ccc{$\mathrm{w}$是公式\ref{TF:weighted:sum}中的权重}\label{w_i:definition}
    \\
    \KwRet{$\lambda s.$\FuncSty{f($\mathrm{w,k,n}$)}$(s)/\binom{n}{k}$} \ccc{$\position$的总共可能取值有$\binom{n}{k}$个}
}

\ccc{下面的函数应该被记忆化,以避免对重复的w,\paramOneInF,和\paramTwoInF{}组合重复计算}\\

\SetKwFunction{FMain}{f}
\SetKwProg{Fn}{Function}{}{}
\Fn{\FMain{$\mathrm{w,\paramOneInF,\paramTwoInF}$}\emph{\ccc{给定w,用公式\ref{f:recursive:equation}计算并返回$\mathrm{f_{\paramOneInF,\paramTwoInF}}$}}}{ 
    \uIf{$\mathrm{\paramOneInF>\paramTwoInF}$}{
        \KwRet{$\mathrm{\lambda s.0}$}
    }
    \uIf{$\mathrm{\paramOneInF=0}$}{
        \KwRet{$\mathrm{\lambda s.\mathbf{1}_{s=0}}$}
    }
    \KwRet{$\lambda s.$\FuncSty{f($\mathrm{w,\paramOneInF,\paramTwoInF-1}$)}$\mathrm{(s)+}$\FuncSty{f($\mathrm{w,\paramOneInF-1,\paramTwoInF-1}$)}$\mathrm{(s-w_{\paramOneInF}\paramTwoInF)}$}
}
\end{algorithm}


我们用$f_{\paramOneInF,\paramTwoInF}(s)$表示满足$1\le \position_1< \ldots < \position_{\paramOneInF}\le \paramTwoInF$且$\sum_{j=1}^{\paramOneInF}w_j\position_j=s$的情况下,$\position_1,\ldots,\position_{\paramOneInF}$的可能取值数量。那么我们要解决的问题就是求出$f_{k,n}(s)$。
作为基础情况,
(1)~$\paramOneInF>\paramTwoInF$时$f_{\paramOneInF,\paramTwoInF}(s)=0$,因为$\position_{\paramOneInF}<\paramOneInF$不可能成立。
(2)~$f_{0,\paramTwoInF}(s)=\mathbf{1}_{s=0}$,其中$\mathbf{1}$是指示函数,因为只有空序列$\langle\rangle$是合法的,并且$\sum_{j=1}^0w_j\position_j=0$。
对于所有$\paramTwoInF\ge \paramOneInF>0$,$f_{\paramOneInF,\paramTwoInF}(s)$的可能取值数有两种情况。(1)~如果$\position_{\paramOneInF}\le \paramTwoInF-1$,根据定义,这个数等于$f_{\paramOneInF,\paramTwoInF-1}(s)$。(2)~如果$\position_{\paramOneInF}=\paramTwoInF$,$s$的可能数量等于满足$\position_{\paramOneInF-1}\le \position_{\paramOneInF}-1= \paramTwoInF-1$和$\sum_{j=1}^{\paramOneInF-1}w_j\position_j=(\sum_{j=1}^{\paramOneInF}w_j\position_j)-w_{\paramOneInF}\position_{\paramOneInF}=s-w_{\paramOneInF}\paramTwoInF$的情况下$\position_1,\ldots,\position_{\paramOneInF-1}$的可能取值数,也就是$f_{\paramOneInF-1,\paramTwoInF-1}(s- w_{\paramOneInF}\paramTwoInF)$。
总的来说,我们有
\begin{equation}\label{f:recursive:equation}
f_{\paramOneInF,\paramTwoInF}(s)=\left\{
\begin{array}{lc}
0 & \quad\paramOneInF>\paramTwoInF\\
\mathbf{1}_{s=0} & \quad\paramOneInF=0\\
f_{\paramOneInF,\paramTwoInF-1}(s)+f_{\paramOneInF-1,\paramTwoInF-1}(s-w_{\paramOneInF}\paramTwoInF) & \quad\mathrm{其它情况}
\end{array}
\right.
\end{equation}
在解出$f_{k,n}$之后,我们得到了对于每种失败测试用例的相对位置的$\sum_{i=1}^m\TF_i$的\distribution{}。
根据对称性,$k!$种相对顺序都有相同的概率取到,因此我们只需这些\distributions{}的平均,就可以得到对所有测试用例执行次序的$\sum_{i=1}^m\TF_i$的\distribution{}。
最后,我们将$\sum_{i=1}^m\TF_i$的\distribution{}转换为\APFD{}的\distribution{}。


接下来我们更详细地描述算法~\ref{algo:distribution:APFD}。
输入是测试用例的数量$n$,缺陷数$m$,以及 \mappingMatrix{}$M$。
主函数\FuncSty{PMF}调用\FuncSty{PMF_sum}来获得 $\sum_{i=1}^m\TF_i$ 的\distribution{},并将其转换为\APFD{} 的\distribution{}。
函数\FuncSty{PMF\_sum}枚举$k$个失败测试用例的所有相对顺序$\relativeOrder$,调用\FuncSty{PMF\_rorder($\mathrm{\relativeOrder}$)}来获得每个相对顺序的$\sum_{i=1}^m\TF_i$的\distribution{}。并对这些\distributions{}进行平均,得到对所有(相对)执行次序的$\sum_{i=1}^m\TF_i$的分布。
函数\FuncSty{PMF\_rorder($\mathrm{\relativeOrder}$)}计算公式(\ref{TF:weighted:sum})中的权重$w$,调用 \FuncSty{f(w,k,n)}以得到对于$w$的$f_{k,n}$,并将其转换为$\sum_{i=1}^m\TF_i$的\distribution{}。

我们最后讨论算法~\ref{algo:distribution:APFD}的时间复杂度和实践中的性能。
算法的主要的运行成本来自于计算函数\FuncSty{f}。
因为有$O(k!)$种不同的$w$和$0\le \paramOneInF\le k,\paramOneInF\le \paramTwoInF\le n$,我们要为$O(nk\multi k!)$种不同的输入计算\FuncSty{f}。
通过记忆化,对每种输入只需计算一次\FuncSty{f}。
每次计算需要$O(nm)$,因为$|\mathrm{support}(f_{\paramOneInF,\paramTwoInF})|=O(nm)$(对于每个$1\le i\le m$,我们有$1\le \TF_i\le n$)。
 因此,对所有输入计算\FuncSty{f}的成本是$O(n^2mk\multi k!)$。
算法中的其他成本低于\FuncSty{f}的成本;因此,算法~\ref{algo:distribution:APFD}的总体时间复杂度为$O(n^2mk\multi k!)$。

\newcommand{\mysubsubsection}[1]{\noindent\textbf{#1:}}
\mysubsubsection{实现}
虽然自上而下的递归使算法展示起来更清楚,但为了获得更好的性能,我们的实现使用自下而上的动态规划来计算\FuncSty{f}。
我们的实现只使用了\CPPLine{}行的C++代码。

\mysubsubsection{数据集}
我们使用拥有最多Java项目的RTP数据集~\cite{Peng2020IRTCP}对算法进行评估。
在这个数据集中,每个测试用例是一个测试类,每个\class{}是一个Maven模块~\cite{2020Maven}。
该数据集有2980个\job{},其中2906个(98\%)有$k\le 10$。
对于每个$k\le 10$,我们从数据集中选择具有最大测试用例数量($n$)的\job{}。
我们还合成了一个有2118(数据集中最大的测试用例数)个测试用例、10个失败的虚拟\job{}。
我们在所选的\job{}上使用了\mappingAllToOne{}和 \mappingOneToOne{}的\mappingMatrices{}。

\begin{table*}[t]
\centering
\vspace{-1em}
\caption{对于10个最大的\job{}~\cite{Peng2020IRTCP}和一个合成\job{} (TSmax),测试用例的数量, 失败的数量,算法运行时间(单位ms),和Jensen-Shannon(JS)距离}
\label{tab:numbersforjobs}
\begin{tabular}{|l||r|r|r|r||r|}
\hline
\multicolumn{1}{|c||}{\textbf{Test}} & \textbf{\#Tests} & \textbf{\#Failures} & \multicolumn{2}{c||}{\textbf{Runtime [ms]}}& \textbf{Jensen-Shannon}  \\
\multicolumn{1}{|c||}{\textbf{suite}} & \multicolumn{1}{c|}{\textbf{$(n)$}} & \multicolumn{1}{c|}{\textbf{$(k)$}} & \textbf{\mappingAllToOne} &\textbf{\mappingOneToOne} & \multicolumn{1}{c|}{\textbf{distance} (\S3.2.2)} \\
\hline
TS1 & 2118 & 1  & 513& 505 & 0.0000\\
TS2 & 1986 & 2   & 563& 629& 0.0005\\
TS3 & 2080 & 3   & 617& 871& 0.0003\\
TS4 & 1929 & 4   & 680& 1147& 0.0004\\
TS5 & 1795& 5&  731&1408& 0.0006\\
TS6 & 339& 6&  627&732& 0.0040\\
TS7 & 465& 7&  678&756& 0.0034\\
TS8 & 813& 8 & 829&2009&  0.0023\\
TS9 & 52& 9&  1496 &1846& 0.0442\\
TS10& 161 & 10&  10989 &27095& 0.0150\\
\hline
TSmax & 2118& 10&  32801 &242400& 0.0011\\
\hline
\end{tabular}
\end{table*}


\mysubsubsection{评估}
如\Table~\ref{tab:numbersforjobs}所示,对于所有真实的\job{},代码在30秒内运行结束(在一台普通的笔记本电脑上);对于合成的\job{},在\mappingAllToOne{}和\mappingOneToOne{}的情况需要更多时间,但仍分别只用了33秒和4分钟。

\section{特殊情况下\APFD{}的\Distributions{}}

我们在Section~\ref{sec:introduction}中提到,最近的RTP研究使用了真实的失败和缺陷,使用了两种\mappingMatrices{}:\mappingAllToOne{}和\mappingOneToOne{}。
我们接下来讨论这两种常用情况下的\APFD{}的\distributions{}。

\noindent\textbf{\allToOneSec{} \mappingAllToOne{}:}
我们首先为\mappingAllToOne推导出\APFD{}的\distribution{}。
在这种情况下,$m=1$,$k\ge 1$,在公式~(\ref{TF:weighted:sum})中$w_1=1,\forall j>1.w_j=0$。
因此,对于$\paramOneInF>1$,递归公式~(\ref{f:recursive:equation})变成$f_{\paramOneInF,\paramTwoInF}(s)=f_{\paramOneInF,\paramTwoInF-1}(s)+f_{\paramOneInF-1,\paramTwoInF-1}(s)$,这与帕斯卡三角形形式相似。
这一观察提示我们,对于\mappingAllToOne{}的 \APFD{}的\distribution{}可能有一个带有二项式系数的封闭解。

\begin{mytheorem}[\mappingAllToOne{}\mappingMatrixThe{}情况下\APFD{}的\distribution{}]\label{distribution-apfd-one2one}
\begin{equation}\label{formula-distribution-apfd-one2one}
    \Prob{\APFD{}= 1-\frac{s}{n}+\frac{1}{2n}}=\frac{\binom{n-s}{k-1}}{\binom{n}{k}},s\in\{1,2,\ldots,n-k+1\}
\end{equation}
\end{mytheorem}
\begin{proof}
对于\mappingAllToOne{},\APFD{}值仅取决于$\TF_1$,这本质上是公式~(\ref{\TF:weighted:sum})中的$\position_1$。
对于$1\le s\le n-k+1$,只要$s=\position_1<\ldots<\position_k\le n$,$\TF_1=s$就成立。
为了满足这个条件,我们只需要在位置$s$之后选择$k-1$个位置。
因此,在$\binom{n}{k}$种从$n$个位置中选$k$个位置的方法中有$\binom{n-s}{k-1}$个满足条件,所以$\Prob{\TF_1= s}=\inlinefrac{\binom{n-s}{k-1}}{\binom{n}{k}}$,公式(\ref{formula-distribution-apfd-one2one})可以由此直接得出。
\end{proof}

通过\formula{formula-distribution-apfd-one2one},我们可以用$O(n)$的时间来计算\APFD{}在\mappingAllToOne{}情况下的\distribution{}。
我们可以从$\binom{k-1}{k-1}=1$开始,用递推公式$\binom{n'+1}{k-1}=\frac{n'+1}{n'-k+2}\binom{n'}{k-1}, n'\ge k-1$,迭代计算所需的二项系数,得到$\binom{n}{k}=\frac{n}{k}\binom{n-1}{k-1}$。

\noindent\textbf{\oneToOneSec{} \mappingOneToOne{}:} 我们接下来考虑\mappingOneToOne{}下\APFD{}的\distribution{}。
在这种情况下,$m=k$,每个失败的测试用例都会发现一个不同的缺陷,所以对于每种$k$个失败测试用例的相对顺序,在公式(\ref{TF:weighted:sum})中有$\forall j.w_j=1$。
因此,运行算法~\ref{algo:distribution:APFD}并对$w$进行记忆化,复杂度变为$O(n^2k^2+k!)$。
$k!$的项来自于遍历所有的相对顺序。
如果我们事先检查\mappingMatrix{}是否为\mappingOneToOne{},我们就可以避免$k!$的项,所以复杂度是$O(n^2k^2)$。

此外,考虑到当$\forall j.w_j=1$时的公式(\ref{f:recursive:equation}),$f_{k,n}$本质上建模了这样的问题:\emph{“计算将整数$s$拆分成$k$个$\{1,2,\ldots,n\}$中不同的整数相加的情况数量”}。
具体来说,$f_{\paramOneInF,\paramTwoInF}(s)$可以被看作将$s$拆分成$\paramOneInF$个$\{1,2,\ldots,\paramTwoInF\}$中不同整数相加的情况数量。
我们有$f_{\paramOneInF,\paramTwoInF}(s)=f_{\paramOneInF,\paramTwoInF-1}(s)+f_{\paramOneInF-1,\paramTwoInF-1}(s-\paramTwoInF)$,因为拆分出的第$\paramOneInF$个整数或是小于$\paramTwoInF$或是等于$\paramTwoInF$,这两种情况分别对应$f_{\paramOneInF,\paramTwoInF-1}(s)$和$f_{\paramOneInF-1,\paramTwoInF-1}(s-\paramTwoInF)$。
据我们所知,这个问题没有已知的封闭解。
考虑到在我们的评估数据集中,99.8\%(2975/2980)的测试场景有$n^2k^2< 10^9$,$O(n^2k^2)$算法对于几乎所有实际情况是足够高效的。

\noindent\textbf{近似解:} 此外,我们可以通过忽略“不同数字”的约束来近似计算\distribution{},即:\emph{“计算将整数$s$拆分成$k$个$\{1,2,\ldots,n\}$中整数相加的情况数量”}。
这个问题有一个很简洁的生成函数$(x+x^2+\ldots+x^n)^k$,其中$x^s$的系数是拆分的数量~\cite{stanley2011enumerative}。
\begin{equation}\label{approximation:APFD:generating:function:coefficient}
    \sum_{i=0}^{\lfloor \frac{s-k}{n}\rfloor}\binom{k}{i}(-1)^i\binom{s-ni-1}{k-1}
\end{equation}
我们可以用两种不同的算法来计算这些系数。
第一种算法首先用帕斯卡尔三角法预计算二项式系数,然后用公式~(\ref{approximation:APFD: generating:function:coefficient})计算所有系数。
第一步需要$O(nk^2)$,因为$s-ni-1\le nk$和$i\le k$。
第二步需要$O(nk^2)$,因为每个$O(nk)$的系数需要$O(k)$来计算($\lfloor\frac{s-k}{n}\rfloor\le k$)。
因此,第一个算法的总体时间复杂度为$O(nk^2)$。
第二种算法直接用快速傅里叶变换~\cite{brigham1988fast}计算生成函数,其首先将$x+x^2+\ldots+x^n$转换为点值表示,将每个点值计算为第$k$次方,然后插值得到系数。
第二个算法需要$O(nk\log(nk))$,因为多项式的长度是$O(nk)$。
比较这两个算法的时间复杂度,当$k$与$n$相比很小时(即$k-\log k<\log n$),第一种算法更好,否则第二种算法更好。

我们用准确的\distribution{}和近似得到的\distribution{}之间的Jensen–Shannon(JS)距离~\cite{endres2003new}来评估我们的近似解。我们在与章节~\ref{sec:dpalgorithm}中相同的真实\job{}中评估了近似解。如表~\ref{tab:numbersforjobs}所示,我们的近似解具有较小JS距离,其中JS距离最大只有\Num{0.0442,($n=52,k=9$)}。

\section{\APFDc{}的\Distribution{}}

\APFDc{}的\distribution{}比\APFD{}的\distribution{}更复杂,因为即使是最简单的\mappingAllToOne{}\mappingMatrix{},\APFDc{}的可能值的数量可以是$\Omega(2^n)$。
例如,考虑$n$个测试用例,代价分别为$1,2,4,\ldots,2^{n-1}$,只有一个测试用例失败并且 \detects{}唯一的缺陷。
 \APFDc{}值取决于在执行次序中位于失败之前的测试用例的代价之和。
$2^{n-1}$种不同的测试用例的集合可以在失败之前,并且每一个集合都有不同的代价总和。
即使是对于\Figure~\ref{fig:example}中的例子,\APFDc{}(33)的支持集大小也比\APFD{}(8)大得多。
