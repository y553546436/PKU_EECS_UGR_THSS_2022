\chapter*{\bfseries 摘要}

回归测试是一项重要的测试活动,其通过运行测试用例集中的测试用例来检查软件的变化,以告知开发人员这些变化是否会导致测试失败。
测试用例排序(Regression Test Prioritization,简称RTP)旨在通过对测试集进行排序,使可能失败的测试提前运行,从而更快地通知开发人员。
许多RTP已经被提出,并经常将随机测试用例排序(简称随机RTP)作为基线算法进行比较。对一个有$n$个测试用例的测试用例集,随机 RTP可能产生$n!$个不同的测试用例执行次序。现有的研究工作在比较时采用从$n!$个不同测试用例执行次序中随机抽样的方法。
然而,目前还没有对随机RTP的理论分析。

我们提出了第一个对随机RTP的理论分析,为其在RTP研究中常用的指标和场景中推导出概率质量函数和期望值。
利用我们的分析,我们重新审视了一些引用量最高的RTP论文,发现其中的一些有关随机RTP的结果由于抽样不足等可能原因并不符合理论结果。
未来的RTP研究可以利用我们的分析,不需要使用随机抽样,而是使用我们的简单公式或算法,与随机RTP进行更精确的比较。

\bigskip
\bigskip

关键词: 测试用例排序,随机,理论分析

\chapter*{\bfseries ABSTRACT}

%{\parindent0pt

Regression testing is an important activity to check software changes by running the tests in a test suite to inform the developers whether the changes lead to test failures.
Regression test prioritization (RTP) aims to inform the developers faster by ordering the test suite so that tests likely to fail are run earlier.
Many RTP techniques have been proposed and are often compared with the random RTP baseline by sampling some of the $n!$ different test-suite orders for a test suite with $n$ tests.
However, there is no theoretical analysis of random RTP. 

We present such an analysis, deriving probability mass functions and expected values for metrics and scenarios commonly used in RTP research.
Using our analysis, we revisit some of the most highly cited RTP papers and find that some presented results may be due to insufficient sampling.
Future RTP research can leverage our analysis and need not use random sampling but can use our simple formulas or algorithms to more precisely compare with random RTP.

\bigskip
\bigskip

KEY WORDS: Regression Test Prioritization, Random, Analysis
%}