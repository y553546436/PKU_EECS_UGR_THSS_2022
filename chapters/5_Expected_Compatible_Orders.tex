\chapter{考虑所有\Compatible{}测试用例执行次序$\consecutivetorders$时的期望值}

在这一节中,我们考虑$\consecutivetorders$中\APFD{}和\APFDc{}的期望值。
如第~\ref{sec:preliminaries}节所定义的,\cOmpatible{}测试用例执行次序不交错执行来自不同\class{}的测试用例。
与$\alltorders$类似,我们首先为$\consecutivetorders$证明一个有用的引理。

\begin{mylemma}\label{lemma:probability_precede_any:compatible}
对于每个缺陷$i$, (注意如果$t\notin T_i$, $C(t)$可能有另一个$t'\in T_i$) 
\begin{equation}\label{probability_precede_any:compatible}
\forall t\notin T_i.\Probn{t< t_{\TF_i}}=\Probn{\forall t'\in T_i.t< t'}=\left\{
\begin{array}{lc}
\frac{1}{|\classSet_i|(|T_{i,\classOfTest{t}}|+1)} & \quad\classOfTest{t}\in \classSet_i\\
\frac{1}{|\classSet_i|+1} & \quad\classOfTest{t}\notin \classSet_i
\end{array}
\right.
\end{equation}
\end{mylemma}
\begin{proof}
对于$\classOfTest{t}\in \classSet_i$的情况,为了使$t\notin T_{i,\classOfTest{t}}$排在所有能\detect{}缺陷$i$的测试用例之前,有两个条件必须成立。
首先,在$\classSet_i$中的所有\class{}中,$\classOfTest{t}$必须是第一个。根据对称性,$\classSet_i$中的每个\class{}都可以以相同的概率$\frac{1}{|classSet_i|}$成为第一个。
第二,$t$必须先于$T_{i,\classOfTest{t}}$的所有测试用例,(与引理~\ref{lemma:probability_precede_any}类似)以$\frac{1}{|T_{i,\classOfTest{t}}|+1}$的概率成立。
这两个条件是独立的,因为它们分别是关于\class{}的顺序和\class{}内的测试用例顺序,而这些顺序是相互独立的。
因此,$t$先于第一个\detect{}到缺陷$i$的测试用例的概率是$\frac{1}{|\classSet_i|(|T_{i,\classOfTest{t}|+1)}$。

对于$\classOfTest{t}\notin \classSet_i$的情况,只有一个条件——$\classOfTest{t}$先于$\classSet_i$中的所有\class{},就能使$t$先于第一个\detect{}到$i$的测试用例,这(与引理~\ref{lemma:probability_precede_any}相似)以概率$\frac{1}{|\classSet_i|+1}$发生。
\end{proof}

\begin{mytheorem}[$\consecutivetorders$中\APFDc{}的期望值]
\begin{equation}\label{expected_apfdc_no_interleaving}
\begin{array}{rl}
\En{\APFDc}=1-\frac{1}{m\multi \Cost{T}}\sum_{i=1}^m\Bigg(&\frac{\sum_{\testClassInFormula \notin \classSet_i}\Cost{T_\testClassInFormula}}{|\classSet_i|+1}+\\
    &\frac{1}{|\classSet_i|}\sum_{\testClassInFormula \in \classSet_i}\left(\frac{\Cost{T_\testClassInFormula \setminus T_{i,\testClassInFormula}}}{|T_{i,\testClassInFormula}|+1}+\frac{\Cost{T_{i,\testClassInFormula}}}{2|T_{i,\testClassInFormula}|}\right)\Bigg)
\end{array}
\end{equation}
\end{mytheorem}
